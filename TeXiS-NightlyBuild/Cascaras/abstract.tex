%---------------------------------------------------------------------
%
%                      abstract.tex
%
%---------------------------------------------------------------------
%
% Contiene el cap�tulo del resumen en ingl�s.
%
% Se crea como un cap�tulo sin numeraci�n.
%
%---------------------------------------------------------------------

\chapter{Abstract}
\cabeceraEspecial{Abstract}

% \begin{FraseCelebre}
% \begin{Frase}
% Perfection is achieved, not when there is nothing more to add, 
% but when there is nothing left to take away.
% \end{Frase}
% \begin{Fuente}
% Antoine de Saint-Exup�ry
% \end{Fuente}
% \end{FraseCelebre}


\section*{Game engine and editor 2D for non-programmers}

A game engine is a development environment that provides tools for creating video games. 
These tools allow the developer to avoid implementing a large amount of functionality and focus
more on the game's development. Some examples of functionalities that engines provide include 
graphic rendering, physics engine, audio system, player input management, resource management, 
network management, etc.

In addition to the mentioned features, game engines can include an editor. Editors are visual 
tools aimed at communicating the developer's actions to the engine. Therefore, they are part of 
the engine's development environment. It is worth noting that game engine editors often have a 
learning curve, especially for those unfamiliar with the specific engine or game development in general. 
However, once developers become familiar with the tools, they can significantly accelerate the 
game creation process and improve productivity.

This represents a significant advantage for experienced developers, but engines like Unity or 
UnrealEngine can be overly complex for people without programming experience, even if their goal 
is to create simple 2D games. A very useful tool to solve this problem is visual programming. 
This type of programming allows users to create logic by manipulating graphical elements instead 
of exclusively specifying them in text. Unity has its Unity Visual Scripting and UnrealEngine
has Blueprints.

The engine for this final degree project consists of a self-sufficient 2D game development environment. 
This means that it will allow managing game resources, scenes, and interactive elements. Additionally, 
it will support the creation of behaviors through visual programming based on nodes, the game's execution 
in the editor, and the creation of final game executables for distribution.

\endinput
% Variable local para emacs, para  que encuentre el fichero maestro de
% compilaci�n y funcionen mejor algunas teclas r�pidas de AucTeX
%%%
%%% Local Variables:
%%% mode: latex
%%% TeX-master: "../Tesis.tex"
%%% End:
