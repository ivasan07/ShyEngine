%---------------------------------------------------------------------
%
%                      resumen.tex
%
%---------------------------------------------------------------------
%
% Contiene el cap�tulo del resumen.
%
% Se crea como un cap�tulo sin numeraci�n.
%
%---------------------------------------------------------------------

\chapter{Resumen}
\cabeceraEspecial{Resumen}

% \begin{FraseCelebre}
% \begin{Frase}
% Si hubiera tenido m�s tiempo, habr�a escrito una carta m�s corta.
% \end{Frase}
% \begin{Fuente}
% Blaise Pascal
% \end{Fuente}
% \end{FraseCelebre}

\section*{Editor y motor de juegos 2D para no programadores}

El desarrollo de videojuegos es un campo que ha experimentado una evoluci�n significativa a lo largo de los a�os. Inicialmente, los videojuegos eran simples programas dise�ados para entretener a los jugadores, pero con el tiempo, han evolucionado en complejidad y alcance. Hoy en d�a, el desarrollo de videojuegos implica la creaci�n de mundos virtuales interactivos que pueden abarcar una amplia variedad de g�neros y plataformas, desde juegos m�viles hasta t�tulos de alta gama para consolas y PC.

\medskip

En nuestro proyecto, abordamos el desaf�o de simplificar el proceso de desarrollo de videojuegos para personas inexpertas en programaci�n y desarrollo. Esto implica estudiar en profundidad qu� es un videojuego, c�mo se crean, los distintos motores de videojuegos 2D disponibles y sus arquitecturas, as� como las diferentes formas de programar el comportamiento de los elementos del juego. Todo ello con el objetivo de dise�ar un editor y un motor de videojuegos que hagan que la creaci�n de juegos sea accesible y f�cil de entender para aquellos que no tienen experiencia en el campo.

\medskip

Este permite que un grupo m�s amplio de personas pueda crear sus propios juegos y dar rienda suelta a su creatividad sin la barrera de la programaci�n compleja.
\endinput

% Variable local para emacs, para  que encuentre el fichero maestro de
% compilaci�n y funcionen mejor algunas teclas r�pidas de AucTeX
%%%
%%% Local Variables:
%%% mode: latex
%%% TeX-master: "../Tesis.tex"
%%% End:
