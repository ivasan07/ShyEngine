%---------------------------------------------------------------------
%
%                      agradecimientos.tex
%
%---------------------------------------------------------------------
%
% agradecimientos.tex
% Copyright 2009 Marco Antonio Gomez-Martin, Pedro Pablo Gomez-Martin
%
% This file belongs to the TeXiS manual, a LaTeX template for writting
% Thesis and other documents. The complete last TeXiS package can
% be obtained from http://gaia.fdi.ucm.es/projects/texis/
%
% Although the TeXiS template itself is distributed under the 
% conditions of the LaTeX Project Public License
% (http://www.latex-project.org/lppl.txt), the manual content
% uses the CC-BY-SA license that stays that you are free:
%
%    - to share & to copy, distribute and transmit the work
%    - to remix and to adapt the work
%
% under the following conditions:
%
%    - Attribution: you must attribute the work in the manner
%      specified by the author or licensor (but not in any way that
%      suggests that they endorse you or your use of the work).
%    - Share Alike: if you alter, transform, or build upon this
%      work, you may distribute the resulting work only under the
%      same, similar or a compatible license.
%
% The complete license is available in
% http://creativecommons.org/licenses/by-sa/3.0/legalcode
%
%---------------------------------------------------------------------
%
% Contiene la p�gina de agradecimientos.
%
% Se crea como un cap�tulo sin numeraci�n.
%
%---------------------------------------------------------------------

\chapter{Agradecimientos}

\cabeceraEspecial{Agradecimientos}

% \begin{FraseCelebre}
% \begin{Frase}
% A todos los que la presente vieren y entendieren.
% \end{Frase}
% \begin{Fuente}
% Inicio de las Leyes Org�nicas. Juan Carlos I
% \end{Fuente}
% \end{FraseCelebre}

Queremos expresar nuestro sincero agradecimiento a todos aquellos que han contribuido de manera significativa en nuestro desarrollo
como estudiantes. En primer lugar, extendemos nuestro reconocimiento a nuestros respetados profesores,
cuya orientaci�n y sabidur�a han sido fundamentales para guiarnos a lo largo de este proceso acad�mico. Sus conocimientos
compartidos y su apoyo constante nos han permitido crecer y prosperar en este proyecto. Adem�s, deseamos mostrar nuestro agradecimiento
a nuestras familias, cuyo inquebrantable respaldo y �nimo han sido una fuente inagotable de motivaci�n. Su apoyo emocional y comprensi�n
han sido esenciales para superar los desaf�os y celebrar los logros. Nuestro m�s sincero agradecimiento a todos aquellos que han estado
a nuestro lado en este viaje, ayud�ndonos a alcanzar este hito acad�mico.

\endinput
% Variable local para emacs, para  que encuentre el fichero maestro de
% compilaci�n y funcionen mejor algunas teclas r�pidas de AucTeX
%%%
%%% Local Variables:
%%% mode: latex
%%% TeX-master: "../Tesis.tex"
%%% End:
