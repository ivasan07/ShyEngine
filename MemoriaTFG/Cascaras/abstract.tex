%---------------------------------------------------------------------
%
%                      abstract.tex
%
%---------------------------------------------------------------------
%
% Contiene el cap�tulo del resumen en ingl�s.
%
% Se crea como un cap�tulo sin numeraci�n.
%
%---------------------------------------------------------------------

\chapter{Abstract}
\cabeceraEspecial{Abstract}

% \begin{FraseCelebre}
% \begin{Frase}
% Perfection is achieved, not when there is nothing more to add, 
% but when there is nothing left to take away.
% \end{Frase}
% \begin{Fuente}
% Antoine de Saint-Exup�ry
% \end{Fuente}
% \end{FraseCelebre}

The development of video games is a field that has undergone significant evolution over the years. Initially, video games were simple
programs with basic behavior and very limited graphics, which generally provided little entertainment value. Over time, and partly due
to technological advancements, they have considerably increased in complexity and scope. Nowadays, video game development involves the 
creation of interactive virtual worlds that can encompass a wide variety of genres and platforms, from mobile games to high-end titles
for consoles and computers.

\medskip

Hence, the most commonly used and powerful tools for game development today require a notable knowledge base in development and have a
steep learning curve. Therefore, they pose a barrier for those individuals who want to get started in the world of development or who
simply want to create their first simple video game without the necessary knowledge or experience.

\medskip

With this project, we aim to simplify the game development process for individuals inexperienced in programming or development in general.
This involves in-depth study of what a video game is, how they are created, the various 2D game engines available and their architectures,
as well as the different ways to program the behavior of game elements. All of this is done with the goal of designing a game editor and 
engine that make game creation accessible and easy to understand for those who lack experience in the field.

\endinput
% Variable local para emacs, para  que encuentre el fichero maestro de
% compilaci�n y funcionen mejor algunas teclas r�pidas de AucTeX
%%%
%%% Local Variables:
%%% mode: latex
%%% TeX-master: "../Tesis.tex"
%%% End:
