%---------------------------------------------------------------------
%
%                          Conclusiones
%
%---------------------------------------------------------------------

\chapter{Conclusiones}

El objetivo de este TFG era desarrollar un motor de videojuegos 2D para no programadores.
Hemos cumplido con lo propuesto. Nuestro motor le abre las puestas a aquellos desarrolladores
con poca experiencia en programaci�n y a la vez cuenta con la suficiente funcionalidad como
para desarrollar videojueos 2D competentes. 

\medskip

Como aplicaciones pr�cticas, nuestro motor se puede usar en el mundo del desarrollo de videojuegos
indie o incluso a nivel did�ctico.

\medskip

A nivel t�cnico, hemos sacado en conclusi�n una serie de aspectos:

\begin{itemize}

    \item Con la implementaci�n actual, las ventanas de ImGUI no se puede mover fuera de la ventana
    principal de SDL, lo que genera incomodidad en algunas situaciones como al implementar un script,
    donde seguramente sea intersante visualizar la escena o alg�n parametro del editor para tomar 
    decisiones.

    \item Hemos tenido que implementar reflexi�n en C++. Hubiera sido m�s c�modo escoger un lenguaje
    con reflexi�n e incluso nos hubiera dado m�s flexibilidad.

\end{itemize}

Como trabajo futuro pensamos en la siguiente funcinalidad:

\begin{itemize}

    \item Poder lanzar el juego en el propio editor y no en una ventana separada. 
        
    \item Cambiar la implementaci�n actual del editor para poder mover las ventanas de ImGUI fuera
    de la ventana principal de SDL.

    \item A�adir un sistema de animaci�n y dibujado para disminuir la dependenia de herramientas 
    externas.

    \item Abstraer al usuario de rutas de ficheros y directorios conviertiendo los ficheros en
    objetos del motor.

    \item Poder profundizar m�s en el scripting, a�adir funcionalidad m�s compleja que permita
    al usuario una mayor expresividad, por ejemplo, a�adiendo arrays editables desde el editor, 
    creaci�n de clases, recursi�n, temporizadores, corutinas, depuraci�n de los nodos en ejecuci�n.

\end{itemize}