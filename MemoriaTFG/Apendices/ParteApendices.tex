%---------------------------------------------------------------------
%
%                          Parte 3
%
%---------------------------------------------------------------------
%
% Parte3.tex
% Copyright 2009 Marco Antonio Gomez-Martin, Pedro Pablo Gomez-Martin
%
% This file belongs to the TeXiS manual, a LaTeX template for writting
% Thesis and other documents. The complete last TeXiS package can
% be obtained from http://gaia.fdi.ucm.es/projects/texis/
%
% Although the TeXiS template itself is distributed under the 
% conditions of the LaTeX Project Public License
% (http://www.latex-project.org/lppl.txt), the manual content
% uses the CC-BY-SA license that stays that you are free:
%
%    - to share & to copy, distribute and transmit the work
%    - to remix and to adapt the work
%
% under the following conditions:
%
%    - Attribution: you must attribute the work in the manner
%      specified by the author or licensor (but not in any way that
%      suggests that they endorse you or your use of the work).
%    - Share Alike: if you alter, transform, or build upon this
%      work, you may distribute the resulting work only under the
%      same, similar or a compatible license.
%
% The complete license is available in
% http://creativecommons.org/licenses/by-sa/3.0/legalcode
%
%---------------------------------------------------------------------

% Definici�n de la �ltima parte del manual, los ap�ndices

%---------------------------------------------------------------------
%
%                          Resumen EN
%
%---------------------------------------------------------------------

\chapter{Abstract}
\cabeceraEspecial{Abstract}

Video game development is a field that has undergone a significant evolution over the years. Initially, video games were simple programs with basic 
simple programs with basic behavior and very reduced graphics, but over time, and due in part to the evolution of technology, they have increased considerably in complexity and scope. 
considerably in complexity and scope. 

\medskip

An execution engine is a software that provides the necessary technology to develop the gameplay of a video game. It is a set of libraries 
(graphics, audio, physics...) grouped in a coherent way that abstracts the technology to the developer so that he can focus on the development of the video game.

\medskip

Nowadays, video game development is a complex task. On the one hand, it can be carried out from scratch, i.e., programming the technology that will be required by the video game and then 
the videogame will require and then developing it and, on the other hand, using an execution engine that provides that independent technology to the specific needs of the videogame to be developed.
specific needs of the video game to be developed.

\medskip

In some cases, in order to help the development, game editors are incorporated. The editors simplify development by linking the creation of game elements, behavior definition, behavioral
game elements, behavior definition, debugging and generation of executable versions, among other things, in a single visual tool. This 
avoids having to communicate directly with the execution engine through programming. 

\medskip

This is a great advantage to experienced developers, but engines like Unity or Unreal Engine can be too complex for non-programmers, even if your programming experience
programming experience, even if they are aiming for simple 2D games. In addition, they are huge systems that pretend to be used to make any game and have a lot of varied 
any game have a lot of varied functionality and create executable versions with a lot of unnecessary data. If you want to make small and simple games 
and simple games, the toll you pay is very high.

\medskip

This is where our work comes in. The idea is to make an engine with its editor to make small games in which the development experience is
equivalent to that of the editors of larger engines, but that is focused on developing smaller 2D games and is much less of a burden
in execution and in generating executable versions. This will open the doors to our engine to developers with little programming experience or even profiles with no development experience at all.
or even profiles with no experience in videogame development at all.

\bigskip

Keywords: Video game editor, Video game development environment, Video game engine, Visual scripting, Node programming
%---------------------------------------------------------------------
%
%                          Introduccion EN
%
%---------------------------------------------------------------------

%-------------------------------------------------------------------
\chapter{Introduction}
\cabeceraEspecial{Introduction}
\section{Motivation}
\label{cap2:sec:motivation}

The development of video games has been a technical and organizational challenge since its inception. Especially at the technical level, it requires the expertise of computer
computer engineers capable of developing the necessary technology that a video game may require. Development companies have a choice
between developing the technologies needed to develop a video game or obtaining those technologies already developed and starting directly with the development.
development.

\medskip

In the first case, the disadvantage lies in developing these technologies in a way that is dependent on the videogame for which it is being developed, i.e., according to the specific characteristics of the videogame,
in such a way that it cannot be reused for the development of another type of videogame.

\medskip

In the second case, the disadvantage lies in the way of obtaining such technologies, licenses or costs. When developing these technologies, teams must draw the line
teams must draw the line between the technologies to develop a videogame and the development of the videogame itself. This separation
is necessary to ensure the reusability of the technologies.   

\medskip

These technologies are known as execution engines. In general, they usually consist of several libraries dedicated to a specific purpose, such as graphics, audio, audio, video, video, etc. 
such as graphics, audio, physics, input, etc.

\medskip

In some cases, in addition to the runtime engine, engine developers include editors. Editors are tools that simplify
development by communicating the developer's actions to the engine. Some of their key functions include the definition of behaviors, the creation of assets and in-game elements, the 
creation of assets and elements in the game, debugging and generation of executable versions for different platforms.

\medskip

There are different engines on the market with different licenses and features. The most convenient for development are those with an editor, but an editor is very costly to develop.
editor is very expensive to develop, so the engines that provide it are those that have a very large critical mass of use, so that it is worth the development effort.
worth the development effort of the editor. This only happens in generalist engines that allow to realize games of many types and with very good quality.
quality. The price to pay for using these engines is their complexity, both in the use of the editor and in the runtime engine. This is not a problem
if the developed game makes use of all the state-of-the-art features, but it is a problem if you want to make a modest game where the gameplay is put to the test and not so much the technology. 
gameplay and not so much the technology.

\medskip

The alternative to make smaller games is to make use of simpler engines, but usually they don't come with an editor and that lengthens the development process.
This is where our end of degree work comes in. We are going to make an engine with its editor to make small games in which the development experience is equivalent to the experience
development experience is equivalent to that experienced with larger engine editors, but that is focused on the development of much smaller 2D games for 
profiles with little experience in programming or game development in general. This simplifies the use of the editor and also the size of the executable versions made.
executable versions.

\medskip

The features of the engine we want to make are:

\begin{itemize}
    \item \textit{Execute the game from the editor}: This means the possibility of launching the game from the editor without having to make an executable version or to look for the executable file. 
    executable version or to look for the executable file by hand. In addition, it will be possible to print information by console visible from the editor either for debugging or to 
    to check for errors.

    \textit{Self-sufficiency}: It provides functionality to manage resources, create scenes and objects from the editor and allows the creation of final executables of the game for its distribution.
    final executables of the game for its distribution. With this feature, the minimum dependence on external tools is sought. 

    \item \textit{Visual programming}: This is the most important part of our engine. Due to the complexity of some of today's engines for beginners or inexperienced developers. 
    for beginners or inexperienced developers, visual programming is a very useful and intuitive tool to create logic and behavior in the video game. 
    game.
\end{itemize}

%-------------------------------------------------------------------
\section{Goals}
\label{cap2:sec:goals}

The main objective is to develop a self-sufficient 2D videogame engine with integrated editor and node-based visual programming.
In addition, it will be possible to populate and view the progress of the game being developed directly from the editor and generate executable versions. 

\medskip

With this, users will have a tool to develop any type of 2D videogame with an accessible level of complexity and a pleasant and intuitive user experience. 
a pleasant and intuitive user experience.

\section{Tools}
\label{cap2:sec:tools}

%-------------------------------------------------------------------
To begin with, Git has been used as a version control system through the GitHub Desktop application. All the code
implemented has been uploaded to a repository.

\medskip

Link to the repository: \url{https://github.com/ivasan07/ShyEngine}

\medskip

The code has been developed in the Visual Studio 2022 integrated development environment (IDE) and written in C++, and the PDF generation has been carried out with Latex.

\medskip

Finally, we have carried out the task management through the Trello project management system.

%-------------------------------------------------------------------
\section{Work Plan}
\label{cap2:sec:workplan}

Our project will be divided into three main blocks: engine, editor and visual scripting.

\medskip

The work will be divided into five phases: research and planning, initial development, core development, development closure and user testing. 

\begin{itemize}
    \item \textit{Research and planning}: 
    The first phase of the work will consist of researching different game engines and publishers to understand how they work and their different
    architectures. We will also look for libraries that fit the demands of our project to achieve a comfortable and efficient development.
    Finally we will plan the division of work as well as the future continuous integration of each of the parts.

    \item \textit{Initial development}: For this phase, we will develop the core of each project. As for the engine, the first tests of the libraries that will be used in the project will be carried out.
    the first tests of the libraries to be used and the basic game architecture will be implemented. Regarding the editor, we will create the project
    in which the selected graphical interface library will be integrated to test its operation. As for the visual scripting, we will start prototyping the language.
    the language. In addition, a continuous integration process will be carried out as this phase progresses in order to avoid possible incompatibilities in the next phase.
    phase. This will involve the definition of an interchange file format that will be the interface between the editor and the engine. The type of information, its basic
    information, its basic structure and how the assets will be referenced.

    \item \textit{Development core}: During this phase, being the longest one, we will add new functionalities in each of the parts, such as new
    components in case of the editor, system of windows and docking for the editor, etc. We will extend the information exchange between engine and editor so that we will continue to integrate and
    we will continue to integrate and create the visual scripting system in the editor to generate the first scripts and the logic to store and interpret them in the engine.
    in the engine.

    \item \textit{Closing of the development}: With the projects finished and fully integrated, we will continue the development of the functionalities related to the improvement of the user experience.
    related to improving the user experience, especially in the editor, to make it as polished as possible before user testing. This will involve the
    testing/development of some gameplay to detect possible bugs and correct them.

    \item \textit{Testing with users}: Tests will be carried out with users from different contexts: users with programming experience and users with no experience.
    inexperienced users. In this way, we will take advantage of their feedback to fix possible bugs and polish details to improve the user experience.
\end{itemize}


%---------------------------------------------------------------------
%
%                          Conclusiones EN
%
%---------------------------------------------------------------------
\chapter{Conclusions}
\cabeceraEspecial{Conclusions}


