%---------------------------------------------------------------------
%
%                          Parte 3
%
%---------------------------------------------------------------------
%
% Parte3.tex
% Copyright 2009 Marco Antonio Gomez-Martin, Pedro Pablo Gomez-Martin
%
% This file belongs to the TeXiS manual, a LaTeX template for writting
% Thesis and other documents. The complete last TeXiS package can
% be obtained from http://gaia.fdi.ucm.es/projects/texis/
%
% Although the TeXiS template itself is distributed under the 
% conditions of the LaTeX Project Public License
% (http://www.latex-project.org/lppl.txt), the manual content
% uses the CC-BY-SA license that stays that you are free:
%
%    - to share & to copy, distribute and transmit the work
%    - to remix and to adapt the work
%
% under the following conditions:
%
%    - Attribution: you must attribute the work in the manner
%      specified by the author or licensor (but not in any way that
%      suggests that they endorse you or your use of the work).
%    - Share Alike: if you alter, transform, or build upon this
%      work, you may distribute the resulting work only under the
%      same, similar or a compatible license.
%
% The complete license is available in
% http://creativecommons.org/licenses/by-sa/3.0/legalcode
%
%---------------------------------------------------------------------

% Definici�n de la �ltima parte del manual, los ap�ndices

%---------------------------------------------------------------------
%
%                          Resumen EN
%
%---------------------------------------------------------------------

\chapter{Abstract}
\cabeceraEspecial{Abstract}

The development of video games is a field that has undergone significant evolution over the years. Initially, video games were simple programs with basic behavior and very limited
graphics. However, over time, and partly due to the evolution of technology, they have increased considerably in complexity and scope.

\medskip

A game engine is software that provides the necessary technology to develop the gameplay of a video game. It's a coherent set of libraries (graphics, audio, physics, etc.) that 
abstracts technology for the developer, allowing them to focus on the game's development.

\medskip

Nowadays, game development is a complex task. On one hand, it can be done from scratch, meaning programming the technology the game will require and then developing it. On the other
hand, you can use a game engine that provides that technology independently of the specific needs of the game to be developed.

\medskip

In some cases, in order to help the development, game editors are incorporated. The editors simplify development by linking the creation of game elements, behavior definition, behavioral
game elements, behavior definition, debugging and generation of executable versions, among other things, in a single visual tool. This avoids having to communicate directly with the videogame
engine through programming. 

\medskip

This presents a significant advantage for experienced developers, but engines like Unity or Unreal Engine can be overly complex for individuals with no programming experience, even if their
goal is to create simple 2D games. Additionally, these are extensive systems designed to handle a wide range of game types, resulting in a lot of varied functionality and creating executable
versions with a large amount of unnecessary data. If you want to create small and simple games, the overhead can be quite substantial.

\medskip

This is where our work comes into play. The idea is to create an engine with its editor for developing small games, where the development experience is equivalent to that of larger engine
editors. However, it will be focused on smaller 2D game development and will have a much lighter load in terms of execution and generating executable versions. This will open the doors to
our engine for developers with little programming experience or even those with no experience at all in game development.

\bigskip

Keywords: Video game editor, Video game development environment, Video game engine, Visual scripting, Node programming
%---------------------------------------------------------------------
%
%                          Introduccion EN
%
%---------------------------------------------------------------------

%-------------------------------------------------------------------
\chapter{Introduction}
\cabeceraEspecial{Introduction}
\section{Motivation}
\label{cap2:sec:motivation}

The development of video games has been a technical and organizational challenge since its inception. Especially at the technical level, it requires the expertise of computer 
computer engineers capable of developing the necessary technology that a video game may require. Development companies have a choice between developing the technologies needed
to develop a video game or obtaining those technologies already developed and starting directly with the development.

\medskip

In the first case, the disadvantage lies in developing these technologies in a way that is dependent on the videogame for which it is being developed, i.e., according to the 
specific characteristics of the videogame, in such a way that it cannot be reused for the development of another type of videogame.

\medskip

In the second case, the disadvantage lies in the way of obtaining such technologies, licenses or costs. When developing these technologies, teams must draw the line teams must 
draw the line between the technologies to develop a videogame and the development of the videogame itself. This separation is necessary to ensure the reusability of the technologies.   

\medskip

These technologies are known as videogame engines. In general, they usually consist of several libraries dedicated to a specific purpose such as graphics, audio, physics, input, etc.

\medskip

In some cases, in addition to the videogame engine, engine developers include editors. Editors are tools that simplify development by communicating the developer's actions to the engine.
Some of their key functions include the definition of behaviors, the creation of assets and in-game elements, the creation of assets and elements in the game, debugging and generation
of executable versions for different platforms.

\medskip

There are different engines on the market with different licenses and features. The most convenient for development are those with an editor, but an editor is very costly to develop, so 
the engines that provide it are those that have a very large critical mass of use, so that it is worth the development effort. This only happens in generalist engines that allow to realize 
games of many types and with very good quality. The price to pay for using these engines is their complexity, both in the use of the editor and in the runtime engine. This is not a problem
if the developed game makes use of all the state-of-the-art features, but it is a problem if you want to make a modest game where the gameplay is put to the test and not so much the technology. 

\medskip

The alternative to make smaller games is to make use of simpler engines, but usually they don't come with an editor and that lengthens the development process. This is where our end of degree
work comes in. We are going to make an engine with its editor to make small games, aiming to provide a development experience equivalent to that of larger game engine editors. However, our focus
will be on creating much smaller 2D games, catering to individuals with limited experience in programming or game development in general. This simplifies the use of the editor and also reduces the
size of the executable versions produced.

\medskip

The features of the engine we want to make are:

\begin{itemize}
    \item \textit{Self-sufficiency}: It provides functionality to manage resources, create scenes and objects from the editor and allows the creation of final executables of the game for its
    distribution. With this feature, the minimum dependence on external tools is sought. 

    \item \textit{Visual programming}: This is the most important part of our engine. Due to the complexity that some engines today pose for beginner or inexperienced developers, visual programming
    is a very useful and intuitive tool for creating logic and behavior in the video game.

    \item \textit{Execute the game from the editor}: This provides the possibility to launch the game from the editor without having to create an executable version or manually search for the 
    executable file. Additionally, you will be able to print information to the console that is visible from the editor, either for debugging or error checking purposes.
\end{itemize}

%-------------------------------------------------------------------
\section{Goals}
\label{cap2:sec:goals}

The main objective is to develop a self-sufficient 2D videogame engine with integrated editor and node-based visual programming. In addition, it will be possible to populate and view the 
progress of the game being developed directly from the editor and generate executable versions. 

\medskip

With this, users will have a tool to develop any type of 2D videogame with an accessible level of complexity and a pleasant and intuitive user experience. 

\section{Tools}
\label{cap2:sec:tools}

%-------------------------------------------------------------------
To begin with, Git has been used as a version control system through the GitHub Desktop application. All the code implemented has been uploaded to a repository.

\medskip

Link to the repository: \url{https://github.com/ivasan07/ShyEngine}

\medskip

The code has been developed in the Visual Studio 2022 integrated development environment (IDE) and written in C++, and the PDF generation has been carried out with \LaTeX.

\medskip

Finally, we have carried out the task management through the Trello project management system.

%-------------------------------------------------------------------
\section{Work Plan}
\label{cap2:sec:workplan}

Our project will be divided into three main blocks: engine, editor and visual scripting.

\medskip

The work will be divided into five phases: research and planning, initial development, core development, development closure and user testing. 

\begin{itemize}
    \item \textit{Research and planning}: 
    The first phase of the work will consist of researching different game engines and publishers to understand how they work and their different architectures. We will also look for
    libraries that fit the demands of our project to achieve a comfortable and efficient development. Finally we will plan the division of work as well as the future continuous integration
    of each of the parts.

    \item \textit{Initial development}: For this phase, we will be developing the core of each project. Regarding the engine, initial testing of the libraries to be used will be conducted,
    and the basic game architecture will be implemented. As for the editor, a project will be created in which the selected graphical interface library will be integrated to test its 
    functionality. Regarding visual scripting, the language will begin to be prototyped. Additionally, a continuous integration process will be carried out as this phase progresses to
    prevent potential incompatibilities in the next phase. This will involve defining a file exchange format that will serve as the bridge between the editor and the engine. The type of
    information, its basic structure, and how assets will be referenced.

    \item \textit{Development core}: During this phase, being the longest one, we will add new functionalities in each of the parts, such as new components in case of the editor, system
    of windows and docking for the editor, etc. We will extend the information exchange between engine and editor so that we will continue to integrate and we will continue to integrate 
    and create the visual scripting system in the editor to generate the first scripts and the logic to store and interpret them in the engine.

    \item \textit{Closing of the development}: With the projects completed and fully integrated, we will continue to develop features related to improving the user experience, especially
    in the editor, to make it as polished as possible before user testing. This will involve testing/developing a game to identify possible errors and correct them.

    \item \textit{Testing with users}: User testing will be conducted with individuals from different backgrounds: users with programming experience and users without experience. This way,
    we will leverage their feedback to address potential issues and refine details that enhance the user experience.
\end{itemize}


%---------------------------------------------------------------------
%
%                          Conclusiones EN
%
%---------------------------------------------------------------------
\chapter{Conclusions}
\cabeceraEspecial{Conclusions}

In the development of our game engine, despite being a relatively simpler project compared to industry-leading engines and having a team of just three members, we have managed to create 
a tool with a fairly accurate finish and functionality comparable to that of other renowned engines in the industry. Furthermore, we have successfully achieved our goal of making game
development more accessible to individuals with no prior experience in the field, thus expanding the possibilities for game development to a wider audience.

\medskip

However, our engine does have certain limitations, such as the inability to move ImGUI windows outside of the main SDL window, which can be inconvenient in certain situations, especially
when implementing scripts. Additionally, we had to implement reflection in C++, which in hindsight could have been simplified by choosing a language with built-in reflection capabilities.
A higher-level language would have also reduced the need to manually manage memory in the editor project, making its development more comfortable.

\medskip

Looking ahead, we have several areas of improvement and potential development for our engine:

\begin{itemize}
    \item \textit{File path abstraction}: We can work on abstracting file and directory paths by turning files into engine objects. This would simplify resource management and provide more
    user convenience.
    
    \item \textit{Add explanatory texts}: Including text-explanatory boxes in areas that might be confusing to users is a great idea. This would improve the user experience and make it easier
    for users to understand the tool.
    
    \item \textit{Contribution of documentation and tutorials}: As a complement to adding explanatory texts, developing a section with documentation and links to tutorials to facilitate the 
    learning curve is also a valuable enhancement. This would provide users with a comprehensive resource to learn and understand the tool more effectively.

    \item \textit{Improve scripting}: Expanding the scripting capabilities would allow users to express their ideas more fully. This could include adding more complex functionality, such as
    managing editable arrays from the editor, creating classes, recursion, timers, coroutines, and debugging running nodes.
    
    \item \textit{Custom data structures}: Provide users with the ability to create custom data structures and group elements within the scripting system, which would increase flexibility 
    and design options.
    
    \item \textit{Execute in the editor}: Allow users to run their games directly in the editor itself rather than in a separate window, which would streamline the development and testing
    process.
    
    \item \textit{Enhanced usability of entities in the scene window}: Work on improving the usability of entities in the scene window so that users can interact and manage their elements more
    efficiently.
\end{itemize}

In summary, while we have achieved a functional and accessible game engine, we acknowledge that there is always room for improvement and expansion. Our goals for the future include making the 
tool even more user-friendly and adding additional features that enable more comprehensive and versatile game development.